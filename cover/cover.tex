%%%%%%%%%%%%%%%%%%%%%%%%%%%%%%%%%%%%%%%%%
% tungufoss Cover letter
% XeLaTeX Template
% Version 1.0 (30/01/15)
%
% This template was based on Friggeri Resume/CV 
% by Adrien Friggeri https://github.com/afriggeri/CV
% and updated by Helga Ingimundardottir https://github.com/tungufoss/cv
%
% License:
% CC BY-NC-SA 3.0 (http://creativecommons.org/licenses/by-nc-sa/3.0/)
%
% Important notes:
% This template needs to be compiled with XeLaTeX and the bibliography, if used,
% needs to be compiled with biber rather than bibtex.
%
%%%%%%%%%%%%%%%%%%%%%%%%%%%%%%%%%%%%%%%%%

\documentclass{cover} 
\usepackage{lipsum}
%----------------------------------------------------------------------------------------
%	YOUR NAME AND CONTACT INFORMATION
%----------------------------------------------------------------------------------------

%\header{helga}{ingimundardottir}{computational engineer} % Your name and current job title/field
	
\signature{Helga Ingimundard\'{o}ttir}
\addrfrom{Kogunarhaed 1\\ % Address
	Gardabaer IS-210\\
	Iceland
}	
\phonefrom{(+354) 865-1341} % Phone number
\emailfrom{helga85@gmail.com} % Email address
	
%----------------------------------------------------------------------------------------
%	ADDRESSEE AND GREETING/CLOSING
%----------------------------------------------------------------------------------------
	
\greetto{To whom it may concern,} % Greeting text
\closeline{Sincerely,} % Closing text
	
\nameto{} % Addressee of the letter above the to address
	
\addrto{
	Human Resources \\ % To address
	SoundCloud \\
	%123 Pleasant Lane \\
	Berlin, Germany
}	
%----------------------------------------------------------------------------------------
\pagestyle{plain}
\firstName{helga}
\lastName{ingimundard\'{o}ttir}
\workTitle{computational engineer}	

\usepackage{url}
\begin{document}
  
\myletter{	
%----------------------------------------------------------------------------------------
%	LETTER CONTENT
%----------------------------------------------------------------------------------------
My name is Helga Ingimundardottir, and I am writing in reply to your jobs 
listing from your website, seeking to fill the position of XXX. 

\lipsum[1-3]


}{}{}
\end{document}

%----------------------------------------------------------------------------------------

I have been awarded the degree of Magister Scientiarum from University of 
Iceland (UI) in computational engineering, February 2010. Morever, I've been 
awarded the degree of Baccalaureus Scientiarum in mathematics from UI in 
February 2008. Since 2009 I've been enrolled as a Ph.D. student in computer 
science under the guidance of Prof. Thomas Philip Runarsson at UI. Expected 
defence of my thesis is at the latest in May of next year. 

The main focus of the Ph.D. research  is Learning Heuristic Search based on a 
case study within the class of scheduling problems, in particular job-shop and 
flow-shop. The research introduces a supervised framework for the discovery of 
dispatching rules for scheduling problems in an automated way. To this  aim, a 
linear classifier based on logistic regression is trained on randomly generated 
problem instances (labelled with their optimal solutions). In that study the 
features from optimal and suboptimal schedules are learned in a supervised 
fashion, such that a support vector machine (SVM) model can learn how to 
classify `good' from `bad' dispatches.

The advantage of the method is related to the use of experimental date to drive 
the discovery of good rules, instead of relying on hand crafted heuristics. 
Experimental results confirm the validity of the approach w.r.t a given number 
of common single priority dispatching rules. The method of generating training 
data is also shown to be critical for success of the method. The study is 
conducted over various different  problem distributions and moreover 
scalability is also taken into consideration. 

My Ph.D. studies also include research in surrogate models in evolutionary 
optimisation. The aim is to reduce the number of costly fitness evaluations 
needed for search. Ordinal regression, or preference learning, implementing a 
kernel-defined feature space is applied to determine sufficiently accurate 
SVM-based models for evolutionary optimisation. A one-dimensional algorithm has 
been successfully generated, and currently a multi-objective version is 
under-way.

During my Ph.D. studies I was solely responsible for the second year 
undergraduate course Operations Research in the department of Industrial 
Engineering at UI for the spring semesters 2011 and 2012. Along those teaching 
responsibilities I attained a graduate diploma in Teaching Studies for Higher 
Education at master's level from UI School of Education. As a spin-off project 
on my research there, I was granted a teaching grant to implement an innovative 
and more `hands on' approach to tutorials at undergraduate level within the 
School of Engineering and Natural Sciences (SENS) at UI. Hence, I understand 
the need for academic  advisement and  program planning in a non-traditional 
higher education environment. 

I was granted a three year stipend for my Ph.D. studies, hence in the end of 
year 2012 my financial resources were depleted, and I had to seek other 
employment. I have since then been working full-time as a computational 
engineer. As a result, my Ph.D. research has not progressed as much as I had 
hope, due to balancing work, school and social life. But I'm fairly confident 
that I should be able to finish my research and defend my thesis by the end of 
next spring semester. 

My professional experience has been in the banking sector and fish processing 
industry. I am currently an employee at Valka ehf in product development of 
X-Ray guided bone detection and intelligent fish portioning. Valka is in 
partnership with Norwegian University of Science and Technology (NTNU) 
engineering department in three dimensional bone detection, for which  I'm the 
on-site collaborator with NTNU. In addition, Valka has received several 
research grants, e.g., one  with Icelandic Centre for Research (RANNIS) which 
I'm the only employee fully devoted to its research and development. Moreover, 
I'm responsible to fill out all progress reports regarding the grant.

I have always taken active part in university activities and societies. I was 
the Ph.D. student representative in SENS Science Committee in 2011-12. In 
addition, I was one of the founder of the interest group, Arkimedes, of Ph.D. 
and post-doctoral students at SENS in 2011. Previous years, I have been 
President of the Student Organisation for Mathematics and Physics (Stigull, 
2006-2007); Treasurer of local Board of European Students of Technology (BEST 
Reykjavik, 2009-20); and Treasurer of Student Organisation for SENS graduate 
students (Heron, 2009-2011).

In this capacity I've shown leadership qualities and project management skills, 
especially in  consideration of having managed a 80-person undergraduate course 
and re-structuring the course during the semester with respect to the demands 
of the students and under the consideration of novel (w.r.t. norms within SENS) 
teaching methods during my aforementioned teaching studies. 

I believe my eclectic background in applied mathematics, computer science and 
computational and industrial engineering in my work related duties as well as 
counselling and advising students qualifies me for consideration for the 
position of Data Scientist. I look forward in discussing how my skills can be 
of value to SoundCloud. 

Regarding the specific requirements stipulated in your advertisement, then I'm 
most proficient in programming with MATLAB and C\#, and using mex-files or dll, 
respectively, to incorperate C or C++ code within my preferred programming  
language. Although, given my background and past experience with various 
programming languages, I can use Java or C depending on the project's needs. 
For everything else, there is StackOverflow to ease the transition to a new 
language. 

I've been working on linux-based platforms for my Ph.D. research, so for the 
past few years I've mostly been implementing my scripts from the shell (or even 
within my code) in bash. However, my work experience has been with Windows in 
which case I use Cygwin. As for statistcal languages I use either MATLAB or R, 
and all plots/graphics for my thesis is done via the ggplot2 package in R. 

Regarding music and SoundCloud in general, I've had an account (tungufoss) for 
years. I've used the services sporadically, mostly for when I need to submerse 
myself in  code-stints, then SoundCloud is my preferred media player. Mostly 
due to the abundant availability of remixes and mix-tapes. I have to admit, at 
times, I get a bit bored with it due to the fact the randomisation of suggested 
songs are extremely lacking. To clarify, I tend to get obsessed with certain 
songs at any given time. So I pick my favourite song at the start of the day, 
and let SoundCloud's recommendation continue in lieu of a proper play-list. 
However, I noticed that the next `similar' songs were always the same ones 
(even in the same order). After a while the predictability gets boring. I'd 
love to work with SoundCloud, if just for the sole reason to improve my own 
listening experience. But in general, the amount of data you have and given the 
possibilities to learn and improve upon them are endless, it would be a dream 
come true to play with it -- let alone doing that for a living.

Finally, I think Berlin would be a nice change from Reykjavik, Iceland. I 
already have several friends currently living in Berlin and they all highly 
recommend it. So if you are willing to consider me, I'm  more than up for the 
challenge! 