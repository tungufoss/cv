%%%%%%%%%%%%%%%%%%%%%%%%%%%%%%%%%%%%%%%%%
% tungufoss Cover letter
% XeLaTeX Template
% Version 1.0 (30/01/15)
%
% This template was based on Friggeri Resume/CV 
% by Adrien Friggeri https://github.com/afriggeri/CV
% and updated by Helga Ingimundardottir https://github.com/tungufoss/cv
%
% License:
% CC BY-NC-SA 3.0 (http://creativecommons.org/licenses/by-nc-sa/3.0/)
%
% Important notes:
% This template needs to be compiled with XeLaTeX and the bibliography, if used,
% needs to be compiled with biber rather than bibtex.
%
%%%%%%%%%%%%%%%%%%%%%%%%%%%%%%%%%%%%%%%%%

\documentclass{cover} 
\usepackage[utf8]{inputenc}
\usepackage[T1]{fontenc}
\usepackage[english]{babel}

\usepackage{biblatex}
\addbibresource{references.bib}


\usepackage{lipsum}
%----------------------------------------------------------------------------------------
%	YOUR NAME AND CONTACT INFORMATION
%----------------------------------------------------------------------------------------

\signature{Helga Ingimundardóttir}
\addrfrom{Kinnargata 6\\ % Address
	210 Garðabær
}	
\phonefrom{865-1341} % Phone number
\emailfrom{tungufoss@gmail.com} % Email address

%----------------------------------------------------------------------------------------
%	ADDRESSEE AND GREETING/CLOSING
%----------------------------------------------------------------------------------------

\greetto{Dear Hiring Committee,} % Greeting text
\closeline{Sincerely,} % Closing text

\nameto{University of Iceland} % Addressee of the letter above the to address

\addrto{
Sæmundargata 2\\
102 Reykjavík
}	
%----------------------------------------------------------------------------------------
\pagestyle{plain}
\firstName{Dr. Helga}
\lastName{Ingimundard\'{o}ttir}
\workTitle{Ph.D. in Computational Engineering}	

\usepackage{url}
\begin{document}
	
\myletter{	
I am writing to express my interest in the \textit{Assistant Professor in Industrial Engineering position} at the University of Iceland. 
As an Icelandic native speaker, I am confident that my language skills, combined with my educational and research background, and my overall skill set perfectly align with the needs of the department.
		
I hold a Ph.D. in Computational Engineering from the Industrial Engineering, Mechanical Engineering, and Computer Science department at the University of Iceland. My dissertation, titled \citetitle{PhD}, which I defended in the summer of 2016, focuses on performance analysis and preference learning, using job shop scheduling as a case study. It also explores creating simple and easy-to-use practical frameworks using data-driven insights. The dissertation's major area of study is the interplay between problem instances and the proposed algorithm's performance, specifically how variations in problem instances can affect algorithm performance. Understanding this relationship is crucial for accurately comparing different approaches and ensuring meaningful comparisons are made.

I have published several papers based on this work, including the most prominent ones: \cite{isda11,InRu14,Ingimundardottir2018}, and \cite{InRu15a}. The latter was even nominated for the Best Paper Award at the Learning and Intelligent Optimization conference series in 2015. I conducted the research and wrote the papers under the guidance of my thesis advisor and co-author, Prof. Tómas Philip Rúnarsson.

During my master's program in Computational Engineering at the University of Iceland, I had the opportunity to study abroad as an exchange student for a semester at the University of Valenciennes. While there, I conducted my master's project under the guidance of Prof. Sylvain Lalot. Together, we co-wrote a journal publication titled \citetitle{fouling} based on my research findings. I was also fortunate enough to receive a grant from the French Embassy to participate in this exchange program.

I participated in large-scale research projects while at deCODE genetics from 2016-2020, where I was responsible for implementing and maintaining the Oxford Nanopore Technologies (ONT) long-range sequencing analysis pipeline. Working closely with the lab department, I helped decide on protocols and collaborated with the ITO to make necessary changes to cluster and disk architecture, ensuring efficient processing of the large amount of data (6 petabytes over 3 years) while maintaining data integrity. My contributions were integral to the productionalization of the massive dataset and running calculations needed for the research projects published in leading journals on genetics, including \cite{decodeBeyter} as second author and \cite{Holley2021} as third author. I primarily worked on the supplementary associated with those articles and actively aided the principal investigators during the research phase.


Here is a list of selected publications that showcase my research expertise and contributions to the field (as described above):
\printbibliography[heading=none]


Apart from my academic research, I have gained valuable experience in various Icelandic industries. As a computational engineer at Valka, I contributed to the development of generalized fish bone detection algorithms and singlehandedly designed and implemented intelligent fish portioning algorithms for multiple species with fast and accurate calculations.

In addition, as an SQL consultant for AGR Dynamics, I provided customized solutions to meet individual customer needs, implementing data and maintaining databases, and developing custom SQL solutions for specific customer requirements in AGR 5, a web-based supply chain management system.
I also served on the advisory board for grant applications to the Technology Development Fund at RANNIS in 2015 and have resumed my role as a board member in 2023. As a data scientist at CCP Games, I played a leading role in the development of a recommendation engine service for new characters in EVE Online while working for their data department. My responsibilities included creating real-time time-series features based on event logs for feature engineering.

Lastly, at Travelshift, as Head of AI Research, I lead a team of AI consultants on a project that involves optimizing travel plans for vacation packages offered on \href{https://guidetoeurope.com/best-vacation-packages}{GuideToEurope.com} using data-driven approaches. I am responsible for designing and implementing AI algorithms, analyzing data, collaborating with other teams, and staying up-to-date with the latest research trends and advancements in AI technology.

Throughout my career, I have gained extensive experience in various industries and companies, such as CCP Games, Travelshift, and Valka, working in areas such as business intelligence, data analysis, and software development. However, I have come to realize that my true passion lies in teaching and research. Recently, during the spring semester of 2023, I had the opportunity to teach business intelligence, which reignited my desire to engage with students and share my findings with the wider community.

My belief in sharing knowledge and engaging in discourse in my field was reinforced when I worked at Travelshift, where their strict enforcement of NDAs and lack of opportunities to share results outside of patent applications goes against my values. Therefore, I am eager to pursue a career in an open platform where I can freely share my research findings and contribute to the advancement of the field.

In my spare time, I enjoy the creative pursuits of bespoke dressmaking and knitting. While indulging in these hobbies, I often find myself approaching them as optimization problems. However, since these are not typically considered significant business cases, they are often overlooked in research. I believe these hobbies can serve as perfect examples for trying out new research methods and gaining public appeal by contextualizing familiar activities with known buzzwords in the field. Moreover, if needed, these optimization problems can be expanded to a more general business case. For instance, the research problem of dress cutting is essentially the same as cutting steel plates for large manufacturing industries.


Furthermore, I am excited about the proposed collaboration with the Medicine faculty on the \textit{Blóðskimun til bjargar} national health initiative. This opportunity aligns with my long-term goal of contributing to the betterment of society through the use of data and technology. I am confident that my expertise in Industrial Engineering, coupled with my experience in data analysis and software development, will enable me to make meaningful contributions to the project.

Please see the enclosed Research and Teaching Plan for more details on my proposed research and teaching goals.


For more information about my work experience and references, please feel free to contact Tómas Philip Rúnarsson (\email{tpr@hi.is}), my Ph.D. advisor; Guðrún Geirsdóttir (\email{gudgeirs@hi.is}), who mentored me during my teaching studies for higher education; and Valþór Druzin Halldórsson (\email{valthor@ccpgames.com}), my supervisor during my time as a Data Scientist at CCP Games. These individuals can provide further insights into my professional capabilities.


I am excited about the opportunity to contribute to the department's efforts in addressing the challenges of the fourth industrial revolution. With my expertise in the field of Industrial Engineering, I am confident that I can make significant contributions to the development of learning and teaching in this area.
Thank you for considering my application. I look forward to discussing my qualifications further in an interview.

}{
Certificates of Education\\
Curriculum Vitae with a complete list of publications and all scholarly work conducted\\
Outline of proposed research and teaching plans}


\end{document}

