%%%%%%%%%%%%%%%%%%%%%%%%%%%%%%%%%%%%%%%%%
% tungufoss Cover letter
% XeLaTeX Template
% Version 1.0 (30/01/15)
%
% This template was based on Friggeri Resume/CV 
% by Adrien Friggeri https://github.com/afriggeri/CV
% and updated by Helga Ingimundardottir https://github.com/tungufoss/cv
%
% License:
% CC BY-NC-SA 3.0 (http://creativecommons.org/licenses/by-nc-sa/3.0/)
%
% Important notes:
% This template needs to be compiled with XeLaTeX and the bibliography, if used,
% needs to be compiled with biber rather than bibtex.
%
%%%%%%%%%%%%%%%%%%%%%%%%%%%%%%%%%%%%%%%%%

\documentclass{cover} 
\usepackage[utf8]{inputenc}
\usepackage[T1]{fontenc}
\usepackage[icelandic]{babel}


\usepackage{lipsum}
%----------------------------------------------------------------------------------------
%	YOUR NAME AND CONTACT INFORMATION
%----------------------------------------------------------------------------------------

%\header{helga}{ingimundardottir}{computational engineer} % Your name and current job title/field

\signature{Helga Ingimundardóttir}
\addrfrom{Kinnargata 6\\ % Address
	210 Garðabær
}	
\phonefrom{865-1341} % Phone number
\emailfrom{tungufoss@gmail.com} % Email address

%----------------------------------------------------------------------------------------
%	ADDRESSEE AND GREETING/CLOSING
%----------------------------------------------------------------------------------------

\greetto{\textbf{Subject: Data Engineer}} % Greeting text
\closeline{Sincerely,} % Closing text

\nameto{} % Addressee of the letter above the to address

\addrto{
	Human Resources\\ % To address
	XXX 
}	
%----------------------------------------------------------------------------------------
\pagestyle{plain}
\firstName{helga}
\lastName{ingimundard\'{o}ttir}
\workTitle{computational engineer}	

\usepackage{url}
\begin{document}
	
	\myletter{	
		%----------------------------------------------------------------------------------------
		%	LETTER CONTENT
		%----------------------------------------------------------------------------------------
		My name is Helga Ingimundardóttir, and I have a PhD in computational engineering with over 10 years work experience in software development, research and data science.
		
		My doctorate was on hyperheuristcs under the guidance of Prof. Tomas Philip Runarsson at the University of Iceland. The main focus of the study is on Job Shop Scheduling Problems and how to automate the scheduling 
		process using e.g. ordinal regression. Moreover, I inspected ``problem difficulty'' and ``algorithm's footprints in instance space'' -- meaning I investigated how changing assumptions on input data effected algorithmic performance. Moreover, I investigated where in the sequential decision making was the most beneficial effect for efficient learning, in particular using imitation learning (a type of reinforcement learning approach).
		
		Alongside my PhD studies I was a full-time C\# software developer for the start-up Valka (now joined under Marel), where I developed an algorithm for their water-jet cutting machine based on the user's perference on cut patterns, whilst minimizing pinbone (and other outlying bones) and maximizing high-valued portions for various types of white fish.
		
		I later joined AGR Dynamics for a year, where I was a SQL consultant and gained real work experience using MSSQL, a trait that has been quite beneficial in my career. I didn't want to pursue a career in consulting, so once I had defended my PhD, I changed careers back into full-time research at deCODE Genetics. I was at deCODE for about 5 years, where I was in charge of implementing and maintaining the Oxford Nanopore Technologies long range sequencing analysis pipeline.
		I worked closely with the lab department in deciding the protocol from LIMS to work with the downstream analysis. Along with collaborating with the ITO in making necessary changes to our cluster and disk architecture in order to process the exuberant amount of data (roughly 6 petabyte over past three years) in an efficient manner, in terms of computational cost (cpu and gpu hours) and most importantly assuring data integrity.
		
		For the past year I've been a data scientist at CCP Games, although technically under the data analytics department, my daily work was mostly with the data engineering team as my expertise is mid-way on that DA/DE spectrum.
		I worked on a recommendation engine service for new characters playing in EVE Online, using their in-game behavior. This involved developing new real-time time-series features using information from proto-events from either redis or kafka streams. Feature engineering used TimescaleDB extension for pgSQL functions. Moreover, I developed ad-hoc metrics to quantify content quality and engagement to these new recommendation models.

		I hope you will consider my application and see an opportunity for an employee with my academic background and diverse work experience as a strong team member on your team.			
		
	}{}{}
\end{document}

Ég heiti Helga Ingimundardóttir og er doktor í reikniverkfræði með yfir 10 ára starfsreynslu í hugbúnaðarþróun, rannsóknum og gagnaúrvinnslu. 				

Ég byrjaði starfsferil minn sem sumarstarfsmaður hjá Landsbankanum 2007-2009: fyrst hjá notendaþjónustunni, síðar gjaldeyris-, afleiðumiðlun og skuldastýring (GASið) og síðast prófanir. Þar kynntist ég bankaumhverfinu og líkaði vel, en ákvað að snúa mér aftur alfarið að akademíu.

Ég vann að doktorsverkefni í reikniverkfræði hjá Háskóla Íslands 2009-2016 og var þar að rýna  aðferðafræði í reiknigreind og undirstöður gagnagreiningar fyrir mismunandi lærdóms reiknirit, með megináherslu á hvar og hvenær er best að leggja mestan þungan á lærdóm fyrir góðan árangur í vélnámi. 

Samhliða doktorsnáminu mínu vann ég í hugbúnaðarþróun hjá nýsköpunarfyrirtækinu Völku (nú sameinað Marel), þar sem áherslan var að þróa hátækni skurðarvél fyrir fiskvinnslu. Starfið fólst í rannsóknum (með tilheyrandi skýrslugerð til RANNÍS) og tæknilegri útfærslu í hugbúnað vinnslulínurnar. Hjá Völku lærði ég að hugsa út fyrir kassann og prófa mig áfram með að beita brjóstvitsaðferðum til að finna nýjar lausnir á verkefnum.

Ég færði mig síðar til AGR Dynamics sem SQL ráðgjafi í rúmt ár. Þar lærði ég mjög mikilvægt veganesti, sem var góð SQL færni sem ég bý enn að, en stutt ráðgjafa verkefni þótti mér ekki nógu krefjandi og leitaði því aftur í rannsóknir. Ég var hjá Íslenskri erfðagreiningu í raðgreiningarteymi þeirra í rúm 5 ár. Þar var ég að sinna samþættingu rannsókna og gagnareksturs, þar sem starf mitt varð meira í átt að devOps en upphaflega stóð til. Ég lærði að verða mjög öguð á ferlana mína og skjölun þeirra, þannig hægt væri að rekja rannsóknarniðurstöður nákvæmlega í hvaða forrit og tól voru notuð, með hvaða fyrirframgefnum stikum, hrágögnum, forsendum o.s.frv.

Undanfarið ár hef ég verið hjá CCP Games og sinnt gagnagreiningu og innleiðingu nýs gagnagrunns rauntímagagna með næstum-rauntíma úrvinnslu sem talar beint við EVE online tölvuleikinn -- virkni sem þurfti að vinna algjörlega frá grunni. Markmiðið var að greina nýja leikmenn, spá fyrir hvers konar leikja-persónuleika þeir hefðu og koma með ráðleggingar um verkefni sem þau gætu tekist á við til að læra betur á leikinn. Fyrir þetta \textit{recommendation service} þurfti að búa til ótal nýja data features sem tímaraðir sem ég bar ábyrgð á að skilgreina og útfæra. 

Ég vona að þið takið umsókn mína til skoðunar og sjáið tækifæri fyrir starfsmann með minn akademíska bakgrunn og fjölbreytta starfsreynslu sem sterkan liðsmann í ykkar hóp. 
