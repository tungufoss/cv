\documentclass[]{cv} % Add 'print' as an option into the square bracket to remove colors from this template for printing
\begin{document}
	
\header{\Huge Research \& Teaching Plan}{}{Proposal by Dr.\;Helga Ingimundard\'{o}ttir}
\removeaside

\section{Introduction}
I am a highly motivated academic with a strong background in mathematics \& computational engineering and have teaching experience in higher education. 

My proposed research plan involves continuing my work on circumventing expert policies in machine learning and exploring the application of AI to textile design and manufacturing optimization. I plan to incorporate innovative teaching methods and technologies in courses such as optimization, operations research, time series analysis, and real-world applications. In addition, I am committed to engaging with the broader community through partnerships and collaborations with companies, public-facing events, and participation in relevant conferences.

\section{Research Plan}
The research plan I propose is two-fold. The first focuses on continuing my PhD research by exploring the future work I proposed in my thesis and had started prototyping before starting my professional career outside academia. Specifically, I plan to investigate circumventing the expert policy with locally optimized approaches with near-optimality focus for imitation learning. This research is relevant to the field of machine learning and artificial intelligence, and has the potential to contribute to the development of better models and algorithms for decision making in complex environments.

The second avenue of investigation is inspired by my extracurricular passions in textiles and sewing. I believe that a theoretical approach to pattern making (namely a 2D free-form bin packing optimization problem) for fabric can have broad applications in industrial engineering, promoting less waste and better use of materials. I am also interested in exploring how AI can help create bespoke knitting patterns, including adjusting for different body shapes and ease of use. This is a computationally challenging problem, which both borders on art and math. 

The research methods I will use for these investigations include a combination of theoretical analysis, experimentation, and empirical studies. I plan to draw on established theoretical frameworks in machine learning and optimization, as well as develop new frameworks that are tailored to the specific challenges of textile research. For empirical studies, I will reach out to local textile communities and manufacturers to gather data and test my hypotheses.

Anticipated findings and contributions to the field include the development of novel approaches to imitation learning and reinforcement learning that can better handle complex, real-world environments. Additionally, I hope to develop new theoretical frameworks for pattern making and explore the computational challenges of generating bespoke knitting patterns. Finally, by incorporating a focus on reducing waste and streamlining design process my research in textiles has the potential to contribute to broader efforts in engineering for sustainability.

\section{Teaching Plan}
As for teaching, I am passionate about creating a learning environment that is student-centered and fosters critical thinking, problem-solving, and collaboration. My approach to teaching is to provide a strong theoretical foundation while also emphasizing the practical applications of the subject matter. I aim to use real-world examples to help students see the relevance of the material to their future careers.

During my Ph.D. studies, I was responsible for teaching the undergraduate course \textit{Operations Research} for two semesters, and alongside those responsibilities I enrolled in the diploma program for Teaching Studies for Higher Education, where I learned innovative teaching methods, such as project and team-based learning. I completed the diploma in 2012. As a result of the work in that class, I applied and received a Grant for Teaching Development with Associate Professor Benedikt Steinar Magnússon to streamline the method for tutorials in Engineering and Natural Sciences. 

My background in mathematics makes me well-suited to teach courses in optimization, operations research, and time series analysis. Additionally, I believe that courses in real-world applications, such as the one I taught on business intelligence in Spring 2023, are essential for preparing students for success in industry.

For these courses, I would propose syllabus that include clear learning objectives, relevant readings and assignments, and assessments that encourage students to think critically and apply what they have learned. I am also interested in incorporating innovative teaching methods and technologies, such as project-based learning, case studies, and online learning tools.

One technology that I have found particularly useful in my teaching is GitHub Classroom. It provides an efficient way to manage group projects and collaborate on code, while also teaching valuable skills in version control and project management. However, I recognize that there are many other tools and methods that can be just as effective and will keep an open mind in exploring them with my students.

Overall, my goal as a teacher is to inspire and support students in their pursuit of knowledge, while also equipping them with practical skills and the ability to think critically and creatively.


\section{Outreach Plan}
Engaging with the broader community is an essential part of my vision, both within and outside the university. I aim to achieve this by actively participating in public events, lectures, and workshops to share my research interests and knowledge with a broader audience.

I am particularly interested in the DataBeers platform, which is a non-profit and open community whose mission is to become a summit for professionals and students inside the scope of Big Data. This platform is well-known and popular amongst my colleagues, and I believe tying in the university's work would be worth pursuing. Furthermore, it's a great opportunity for graduate students to introduce their work in a relaxed semi-professional environment to the general public.

In addition, I plan to leverage my connections with the companies I have previously worked with, especially CCP Games, to collaborate with them on practical real-world problems that the students could try tackling as they are happy to provide them with a lot of data to analyze.

For the textile optimization projects that I've mentioned earlier, I plan to reach out to local manufacturing companies to collaborate and provide data for cut patterns. Moreover, I will leverage my connections within the hand-knitting community to engage with prominent designers and experts in the field to provide better outreach and expert input.

In addition to these community engagement efforts, I plan to continue attending the conference series Learning and Intelligent Optimisation (LION), which I have found to be incredibly enlightening and relevant to my research interests. This will allow me to stay up-to-date with the latest developments in the field and connect with other experts and researchers in the community.

In summary, my vision is to engage with the broader community through various partnerships and collaborations, public-facing events, lectures, and workshops. I believe this approach will help bridge the gap between academia and industry, and provide students with practical experience to prepare them for their future careers.


\section{Conclusion}
In summary, my proposed research plan is to investigate various optimization techniques to tackle real-world problems, which I believe will contribute to the field of optimization. My teaching plan aims to engage students through innovative teaching methods, incorporating technologies like GitHub Classroom, and participating in public-facing events like Reykjavík DataBeers. Additionally, my recent undergraduate teaching has renewed in me a passion for teaching and helping students learn.

I am excited about the possibility of rejoining the Industrial Engineering department as a faculty member and contributing to your research and teaching efforts. I believe my academic background, research interests, and teaching experience make me a strong candidate for this position. Thank you for considering my application, and I look forward to discussing my qualifications and interests further.

\end{document}