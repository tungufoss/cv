\documentclass[]{cv} % Add 'print' as an option into the square bracket to remove colors from this template for printing
\begin{document}
	
\header{\Huge Research \& Teaching Plan}{}{Proposal by Dr.\;Helga Ingimundard\'{o}ttir}
\removeaside

\section{Introduction}
I am a highly motivated academic with a strong background in mathematics \& computational engineering and have teaching experience in higher education. My research interest are in optimization, heuristics, learning algorithms and real-world problems.

My proposed research plan encompasses four key areas. First, I plan to continue my Ph.D. work on performance analysis and investigate successful learning approaches for optimization. Second, I want to apply artificial intelligence approaches to textile design and manufacturing, including developing new theoretical frameworks for pattern making and exploring the computational challenges of generating bespoke knitting patterns. Third, I am interested in collaborating with the Faculty of Medicine at the University of Iceland on the "Blóðskimun til bjargar" initiative. Finally, I am fascinated by uncovering the hidden truths behind the ground truth data of leading businesses' entertainment ranking systems, with the aim of developing more transparent and inclusive alternatives while being aware of ethical concerns. My research methods will include a combination of theoretical analysis, experimentation, and empirical studies, and I hope to contribute to broader efforts in engineering for sustainability.

I plan to incorporate innovative teaching methods and technologies in courses such as optimization, operations research, time series analysis, and real-world applications. In addition, I am committed to engaging with the broader community through partnerships and collaborations with companies, public-facing events, and participation in relevant conferences. My ultimate goal is to equip my students with the right tools and mentality that are necessary to thrive in the Icelandic workforce while promoting interdisciplinary collaboration and ethical considerations.


\section{Research Plan}
% one 
The research plan I propose is three-fold. The first focuses on continuing my PhD research by exploring the future work I proposed in my thesis and had started prototyping before starting my professional career outside academia. Specifically, I plan to investigate how to circumvent the expert policy with with a near-optimality focus for imitation learning. This research is relevant to the field of machine learning and artificial intelligence, and has the potential to contribute to the development of better models and algorithms for decision making in complex environments.

% two
The second avenue of investigation is inspired by my extracurricular passions in textiles and sewing. I believe that applying a theoretical approach to pattern making, specifically a 2D free-form bin packing optimization problem, can have broad applications in industrial engineering by promoting less waste and better use of materials. Additionally, I am interested in exploring how AI can create bespoke knitting patterns, accounting for different body shapes and ease of use. Although this computationally challenging problem borders on both art and math, it presents an exciting opportunity to explore the intersection of the two fields.

% three
My third research topic focuses on applying machine learning techniques to the field of medical research.
I am currently in contact with Dr. Sigurður Yngvi Kristinsson and Þorvarður Jón Löve fomr the Faculty of Medicine at the University of Iceland, regarding a potential collaboration on their national health project, \textit{Blóðskimun til bjargar}. As part of their study, they are conducting the first population-based screening for monoclonal gammopathy of undetermined significance (MGUS) in Iceland. However, there is a need for an appropriate risk model to identify candidates for early treatment in multiple myeloma (MM). I am excited to propose the use of machine learning techniques on their massive health dataset to help develop such a risk model. With my background in AI and previous work experience at deCODE Genetics, I believe that a collaboration between our departments at the University of Iceland could be fruitful in achieving this goal. This collaboration has the potential to make a significant impact on improving healthcare outcomes for patients.


% fourth 
Additionally, I am fascinated by the hidden truths behind ground truth data of leading businesses' entertainment ranking systems, which is the fourth research area I plan to investigate. At Travelshift, I studied the Google Travel Ranking, which considers various attributes such as ratings and reviews on Google Places, but the exact formula remains undisclosed. Similarly, the BoardGameGeek board game ranking system also has an unknown formula. My goal is to take on the role of Dorothy, uncovering the secrets hidden behind the Wizard of Oz's curtain of secrecy and IP protection that companies often hide from the public. However, ethical concerns arise when sharing an approximation of trade secrets with the general public, which I plan to discuss further with my colleague, Henry Alexander Henrysson, a research specialist at the Center of Ethics at the School of Humanities. Despite these concerns, investigating these rankings can help companies develop more transparent and inclusive alternatives, incorporating alternate viewpoints and biases to promote fairness and inclusivity.


% summary 

The proposed research will use a combination of theoretical analysis, experimentation, and empirical studies, leveraging established frameworks in machine learning and optimization, and developing new ones customized for the unique challenges of textile research. The anticipated results of this work include the development of novel approaches to imitation and reinforcement learning for handling complex real-world environments, the creation of new theoretical frameworks for pattern making, and insights into the computational difficulties of producing bespoke knitting patterns. Moreover, this research has the potential to contribute to broader engineering sustainability efforts by focusing on waste reduction and design streamlining. Additionally, I am excited to be in discussions with the doctors involved in the national health project \textit{Blóðskimun til bjargar}, through such collaborations, I hope to make meaningful contributions to both academia and society at large.


\section{Teaching Plan}
My teaching philosophy centers on creating a student-centered learning environment that promotes critical thinking, problem-solving, and collaboration. I believe in providing a strong theoretical foundation while emphasizing the practical applications of the subject matter. To illustrate the relevance of the material to students' future careers, I use real-world examples in my teaching.

During my Ph.D. studies, I taught the undergraduate course \textit{Operations Research} for two semesters. In addition, I completed the diploma program for Teaching Studies for Higher Education in 2012, where I learned innovative teaching methods such as project and team-based learning. This experience led me to apply for and receive a Grant for Teaching Development with Associate Professor Benedikt Steinar Magnússon from the faculty of physical sciences, where we streamlined the tutorial method in Engineering and Natural Sciences.

Given my background in mathematics, I am well-suited to teach courses in optimization, operations research, and time series analysis. I also believe that courses in real-world applications, such as the one I taught on business intelligence in Spring 2023, are essential for preparing students for success in industry.

	
For my courses, I propose syllabi that include clear learning objectives, relevant readings and assignments, and assessments that encourage critical thinking and practical application of the material. I am also interested in incorporating innovative teaching methods and technologies such as project-based learning, case studies, and online learning tools.

One technology that has been particularly useful in my teaching is GitHub Classroom. It allows for efficient management of group projects and collaboration on code while teaching valuable skills in version control and project management. However, I remain open-minded about exploring other tools and methods with my students.

Overall, my goal as a teacher is to inspire and support students in their pursuit of knowledge, while also equipping them with practical skills and the ability to think critically and creatively.


\section{Outreach Plan}
Engaging with the broader community is a crucial part of my vision, both within and outside the university. I intend to achieve this by actively participating in public events, lectures, and workshops to share my research interests and knowledge with a wider audience.

I am particularly interested in the DataBeers platform, a non-profit and open community focused on Big Data. This platform is popular among my colleagues, and I believe it's worth exploring ways to connect the university's work with this community. Furthermore, it presents an excellent opportunity for graduate students to present their work in a relaxed, semi-professional environment to the general public.

In addition, I plan to leverage my connections with the companies I have previously worked with, particularly CCP Games, to collaborate on practical real-world problems that students could try tackling. They are willing to provide a lot of data for analysis.

For the textile optimization projects I mentioned earlier, I plan to collaborate with local manufacturing companies to provide data for cut patterns. Moreover, I will engage with prominent designers and experts in the hand-knitting community to provide better outreach and expert input.

In addition to these efforts, I plan to continue attending the Learning and Intelligent Optimisation (LION) conference series, which I find incredibly enlightening and relevant to my research interests. This will allow me to stay up-to-date with the latest developments in the field and connect with other experts and researchers.

In summary, my vision is to engage with the broader community through partnerships, collaborations, public-facing events, lectures, and workshops. I believe this approach will help bridge the gap between academia and industry, and provide students with practical experience to prepare them for their future careers. Additionally, I plan to incorporate feedback and adjust my approach based on the needs and interests of the community.


\section{Conclusion}
In summary, I have proposed a research plan that investigates optimization techniques to solve real-world problems, with the aim of contributing to the field of optimization. My teaching plan emphasizes innovative methods and technologies, such as GitHub Classroom and public-facing events like Reykjavík DataBeers. I am enthusiastic about the possibility of rejoining the Industrial Engineering department as a faculty member and believe that my academic background, research interests, and teaching experience make me a strong candidate for this position. I appreciate your consideration of my application and look forward to discussing my qualifications and interests further.





\end{document}