%%%%%%%%%%%%%%%%%%%%%%%%%%%%%%%%%%%%%%%%%
% tungufoss Resume/CV
% XeLaTeX Template
% Version 1.0 (30/01/15)
%
% This template was based on Friggeri Resume/CV 
% by Adrien Friggeri https://github.com/afriggeri/CV
% and updated by Helga Ingimundardottir https://github.com/tungufoss/cv
%
% License:
% CC BY-NC-SA 3.0 (http://creativecommons.org/licenses/by-nc-sa/3.0/)
%
% Important notes:
% This template needs to be compiled with XeLaTeX and the bibliography, if used,
% needs to be compiled with biber rather than bibtex.
%
%%%%%%%%%%%%%%%%%%%%%%%%%%%%%%%%%%%%%%%%%

\documentclass[]{cv} % Add 'print' as an option into the square bracket to remove colors from this template for printing

\addbibresource{references.bib} % Specify the bibliography file to include 
%publications

\begin{document}

\header{helga}{ingimundard\'{o}ttir}{computational engineer} % Your name and 
%current job title/field

%-------------------------------------------------------------------------------
%	SIDEBAR SECTION
%-------------------------------------------------------------------------------

\begin{aside} % In the aside, each new line forces a line break
\section{contact}
%\address{Taeknigardur, \mbox{Office \#217}}{Dunhaga 5}{Reykjavik IS-107}{Iceland}
\address{Helga Ingimundardóttir}{Kinnargata 6}{Gardabaer IS-210}{Iceland}
~
\mobile{(+354) 865 1341}
%\phone{(+354) 525-4704}
~
\email{tungufoss@gmail.com}
%\homepage{https://notendur.hi.is/hei2}{www.hi.is/$\sim$hei2}
\twitter{tungufoss}
\github{tungufoss}
\facebook{helga.ingimundardottir}
\linkedin{helgaingimundardottir}
\orcid{0000-0002-2780-3546}
\section{languages}
Icelandic mother tongue
English fluency
Danish conversational
French conversational
\section{programming}
\entryaside{All-purpose}{C$\#$, C++, Python}
\entryaside{Numerical}{MATLAB}
\entryaside{Statistical}{R, tidyverse}
\entryaside{SQL}{Microsoft SQL Server, PostgreSQL}
\entryaside{Optimisation}{GLPK, Gurobi}
\entryaside{Scripting}{awk, grep, sed, make}
\end{aside}


\section{education}
%-------------------------------------------------------------------------------
%	EDUCATION SECTION
%-------------------------------------------------------------------------------
\begin{entrylist}
%------------------------------------------------
\entry
{2009--2016}
{Ph.D. {\normalfont of Computational Engineering}}
{The University of Iceland, Reykjavik}
{Worked on a doctorate on hyperheuristcs under the guidance of Prof. Tomas 
Philip Runarsson. The main focus of the study is on Job Shop Scheduling 
Problems (JSP) and Flow Shop Problems (FSP) and how to automate the scheduling 
process using e.g. ordinal regression. Moreover, I inspected ``problem 
difficulty'' and ``algorithm's footprints in instance space''. \\
{\boldfont courses in Ph.D. programme:} 
ethics of science and research, communication skills for doctoral students, leadership skills for doctoral students, research plans and applications writing, theoretical statistics, high performance computing A and B  \\
{\boldfont thesis:} entitled \emph{ALICE: Analysis \& Learning Iterative 
Consecutive Executions} is available at \url{http://hdl.handle.net/1946/25337}.}
%------------------------------------------------
\entry
{2010--2012}
{Graduate Diploma {\normalfont of School of Education}}
{The University of Iceland, Reykjavik}
{Teaching Studies for Higher Education.}
%------------------------------------------------
\entry
{2008--2010}
{Masters {\normalfont of Computational Engineering}}
{The University of Iceland, Reykjavik}
{\emph{Detection of Fouling: Effectiveness Ratio Method} \\ 
The dissertation investigated the possibility of using models to detect fouling in a cross-flow heat exchangers, by only using measurements that are attainable in normal operation of the heat exchanger.  The on-line detection of fouling is used by a new and more general method that also takes into account that the input can be varying. \\
The new method finds a threshold for fouling based on the estimate of the steady states of the effectiveness, which is done by applying a wavelet transform since the transform is localised in both time and frequency. \\
The parameters of the method need to be chosen carefully, e.g. compromise between the frequency and time localisation, thus a multiple objective genetic algorithm is implemented for the optimisation. }
%------------------------------------------------
\entry
{2005--2008}
{Bachelor {\normalfont of Mathematics}}
{The University of Iceland, Reykjavik}
{Specialization in Computer Science}
%------------------------------------------------
\end{entrylist}


\section{interests}
%-------------------------------------------------------------------------------
%	INTERESTS SECTION
%-------------------------------------------------------------------------------
{\boldfont professional:} heuristics, artificial intelligence, evolutionary 
computation, global optimisation, statistical learning, machine learning, big data, automation, data visualisation and real world applications 

{\boldfont personal:} knitting, sewing, general arts and crafts, horticulture,  podcasting, internet cats and Russian Blues

\newpage
\removeaside


\section{work experience}

%-------------------------------------------------------------------------------
%	WORK EXPERIENCE SECTION
%-------------------------------------------------------------------------------

\begin{entrylist}
\entry
{2021--}
{CCP Games}
{Reykjavik, Iceland}
{\emph{Data Scientist} for
	\href{https://www.ccpgames.com/}{CCP Games}' data department.\\
	Worked on a recommendation engine service for new characters playing in EVE Online based on their in-game behaviour. This involved developing new real-time time-series features using information from proto-events from either redis or kafka streams. Feature engineering used TimescaleDB extension for pgSQL functions. Moreover, developed ad-hoc metrics to quantify content quality and its engagement to these new recommendation models.
}
%------------------------------------------------
\entry
{2016--2021}
{deCODE Genetics}
{Reykjavik, Iceland}
{\emph{Research Scientist} for
    \href{https://www.decode.com/}{deCODE}'s statistical department.\\
    I was in charge of implementing and maintaining the Oxford Nanopore Technologies long range sequencing analysis pipeline for start of ONT sequencing at deCODE. 
    I worked closely with the lab department in deciding the protocol from LIMS to work with the downstream analysis. Along with collaborating with the ITO in making necessary changes to our cluster and disk architecture in order to process the exuberant amount of data (roughly 6 petabyte over past three years) in an efficient manner, in terms of computational cost (cpu and gpu hours) and most importantly assuring data integrity. 
}
%------------------------------------------------
\entry
{2015--2016}
{AGR Dynamics}
{Reykjavik, Iceland}
{\emph{SQL Consultant} for
    \href{http://agrdynamics.com/}{AGR 5}\\
    AGR 5 is a fully web based solution for supply chain management. AGR 5 
    helps users to visualise sales history and makes order proposals using 
    statistical forecasting. My role for AGR 5 is on the back-end, with data 
    implementation and database maintenance.
}    
%------------------------------------------------
\entry
{2015}
{RANNIS}
{Reykjavik, Iceland}
{\emph{Advisor} in 
\href{http://www.rannis.is/sjodir/rannsoknir/taeknithrounarsjodur/fagrad/}{Technology
 Development Fund}\\
I was on the advisory board that reviews grant applications submitted to 
the  Technology Development Fund at The Icelandic Centre for Research (RANNIS).
The role of the fund is to support R$\&$D in the field of technological 
development aimed at innovation in the Icelandic economy.
Donations in the Technology Development Fund 2004--2014 were a total 8,580 
million ISK, thereof 1,372.5 million ISK for 2015.
}
%------------------------------------------------
\entry
{2013--2015}
{VALKA}
{Kopavogur, Iceland}
{\emph{Computational Engineer} in Research and Development \\
Full time researcher at Valka, which specializes in the development and market\-ing of equipment and automation solutions for the fish processing industry. Valka was the recipient of the Icelandic Innovation Award 2013. \\
Detailed achievements:
\begin{itemize}
\item Designed and implemented an intelligent fish portioning algorithm, based 
on fillet's X-Ray imagery. 
\item Generalised their fish bone detection algorithm in order to analyse more species. Fast calculations, yet sufficiently accurate, for real-world processing plants.
\item Collaborator on three dimensional visualisation of fish bones.
\item Conducted and prepared reports for efficiency tests.
\end{itemize}
}
%------------------------------------------------
\entry
{2007--2009}
{LANDSBANKINN}
{Reykjavik, Iceland}
{
\emph{Summer Intern} at: Testing Department in 2009; Quantitative Research and Trading Support for the FX and Derivatives Sales in 2008; and at Business Support in 2007.
}
%------------------------------------------------
\end{entrylist}
%------------------------------------------------

\section{academic experience}
%-------------------------------------------------------------------------------
%	ACADEMIC EXPERIENCE SECTION
%-------------------------------------------------------------------------------
\begin{entrylist}
    %------------------------------------------------
    \entry
    {2011--2012}
    {University of Iceland, Industrial Engineering Department}
    {Reykjavik Iceland}
    {\emph{Associate Lecturer (i. stundakennari)} in Operations Research \\
        Responsible for the under graduate course Operations Research 
        (\href{https://ugla.hi.is/kennsluskra/index.php?sid=&tab=nam&chapter=namskeid&id=08213020110}{IDN401G}),
         spring semesters 2011 and 2012. \\
        During that period, I restructured the course under the guidance of 
        Gudrun Geirsdottir at School of Education. Moreover, as a result of my 
        efforts in innovating the course design, assessment and evaluations of 
        tutorials, I inspired a fellow teacher in Natural Sciences, and we were 
        awarded a teaching grant for the University for developing our methods 
        further.
    }
    %------------------------------------------------
    \entry
    {2007--2010}
    {University of Iceland, Industrial Engineering Department}
    {Reykjavik Iceland}
    {\emph{Assistant Teacher (i. dæmatímakennari)}, School of Engineering and 
    Natural Sciences \\
        Worked as a tutor during tutorials, correcting and working through 
        handouts for the following under graduate courses: 
        \begin{itemize}
            \item Linear Algebra 		\hfill Autumn~2007
            \item Simulations			\hfill Spring~2008
            \item Operational Research 	\hfill Spring~2008
            \item Calculus IB			\hfill Autumn~2009
            \item Numerical Analysis 	\hfill Spring~2010
        \end{itemize}	
    }
\end{entrylist}   

\section{communication skills}
%-------------------------------------------------------------------------------
%	COMMUNICATION SKILLS SECTION
%-------------------------------------------------------------------------------
\begin{entrylist}
    \entry{2017-2020}{Podcast host, ÍSKISUR}{Alvarpið \& Storytel}{An Icelandic podcast with three friends who read all 47 books in The Legend of the Ice People series by Margit Sandemo. I curated a segment on Internet cats at the end of each episode. Originally published by Alvarpið from 2017-2018 but moved over to Storytel Iceland in March 2019.
}%------------------------------------------------
\entry
{2016}
{Oral Presentation}
{PhD defence 30th of June at Háskóla Íslands, Reykjavík, Iceland.}
{Presented my PhD thesis \emph{ALICE: Analysis \& Learning Iterative 
Consecutive Executions}. 
Opponents: Prof. Edmund Burke and Prof. Kate Smith-Miles.}
%------------------------------------------------
\entry
{2015}
{Oral Presentation}
{9th Int'l Conference on Learning and Intelligent Optimization (LION9)}
{Presented the paper \emph{ Generating Training Data for Supervised Learning 
Linear Composite Dispatch Rules for Scheduling}, Lille, France.}
%------------------------------------------------
\entry
{2012}
{Oral Presentation}
{6th Int'l Conference on Learning and Intelligent Optimization (LION6)}
{Presented the paper \emph{ Determining the Characteristic of Difficult Job 
Shop Scheduling Instances for a Heuristic Solution Method}, Paris,  France.}
%------------------------------------------------
\entry
{2011}
{Oral Presentation}
{\mbox{11th Int'l Conference on Intelligent Systems Design \& Applications (ISDA)}}
{Presented the paper  \emph{Sampling Strategies in Ordinal Regression for 
Surrogate Assisted Evolutionary Optimization}, Cordoba, Spain.}
%------------------------------------------------
\entry
{2010}
{Oral Presentation}
{5th Int'l Conference on Learning and Intelligent Optimization (LION5)}
{Presented the paper \emph{Supervised Learning Linear Priority Dispatch Rules 
for Job-Shop Scheduling}, Rome, Italy}
%------------------------------------------------
\entry
{2010}
{Invited speaker}
{Silisian University, Gliwice, Poland}
{In collaboration with Prof. Waldemar Grzechca at the Silisian University, I was invited to present my Ph.D. research to their faculty.}
%------------------------------------------------
%\entry
%{2009}
%{Poster}
%{11th Int'l Conference on Heat Exchanger Fouling and Cleaning }
%{Presented the paper \emph{Detection of Fouling in a Cross-Flow Heat Exchanger Using Wavelets}, Schladming, Austria.}
%------------------------------------------------
\end{entrylist}
\begin{entrylist}
\entry
{2009}
{Presentation }
{University of Valenciennes and Hainaut-Cambresis, Valenciennes, France}
{As part of the collaboration with Sylvan Lalot, I presented the research faculty at the ENSIAME department at UVHC.}
%------------------------------------------------
\end{entrylist}
\section{grants}
%-------------------------------------------------------------------------------
%	GRANTS SECTION
%-------------------------------------------------------------------------------
\begin{entrylist}
    %------------------------------------------------
    \entry
    {2012}
    {Grant for Teaching Development}
    {Univeristy of Iceland, Kennslumálasjóður}
    {Grant for implementing a new teaching method for tutorials in Engineering 
        and 
        Natural Sciences. Collaboration between Engineering faculty and Natural 
        Sciences faculty. }
    %------------------------------------------------
    \entry
    {2009-2012}
    {Postgraduate Scholarship}
    {University of Iceland Research Fund}
    {Three year stipend for doctoral studies.}
    %------------------------------------------------
    \entry
    {2010}
    {Mobility grant}
    {Fundusz Stypendialny i Szkoleniowy 
        \href{http://www.fss.org.pl/en/about}{(FSS)}} % (e. Scholarship and 
    %training fund between Iceland, Liechtenstein, Norway and Poland)}
    {Mobility grant to visit Silisian University, Gliwice, Poland.}
    %------------------------------------------------
    \entry
    {2009}
    {Postgraduate Scholarship}
    {French Embassy}
    {Awarded to Icelandic students pursuing a Masters degree.}
    %------------------------------------------------
\end{entrylist}

\section{awards}
%-------------------------------------------------------------------------------
%	AWARDS SECTION
%-------------------------------------------------------------------------------
\begin{entrylist}
    %------------------------------------------------
    \entry
    {2015}
    {Nominated for Best Paper award}
    {9th Int'l Conference on Learning and Intelligent Optimization}
    {I had one of three full-paper submission nominated for Best Paper award, 
        on my 
        paper \emph{Evolutionary Learning of Weighted Linear Composite 
            Dispatching 
            Rules for Scheduling}.}
    %------------------------------------------------
    \entry
    {2005}
    {Magna cum laude}
    {The Commercial College of Iceland, Reykavik, Iceland}
    {Awarded for being the top third student in my final year of a 
        Baccalaureate degree.}
    %------------------------------------------------
\end{entrylist}

\section{extracurricular activity}
%-------------------------------------------------------------------------------
%	EXTRACURRICULAR ACTIVITY SECTION
%-------------------------------------------------------------------------------
\begin{entrylist}

    %------------------------------------------------
    \vspace{-9pt}
    \entry{2018-2021}{Board member of Company Union}{deCODE Genetics}{}\vspace{-9pt}
    %------------------------------------------------
    \entry{2016}{Treasurer of Company Union}{AGR Dynamics}{}\vspace{-9pt}
    %------------------------------------------------
    \entry{2014-2015}{Treasurer of Company Union (Salka)}{Valka}{}\vspace{-9pt}
    %------------------------------------------------    
    \entry{2011-2012}{Graduate student representative in Science Committee}{SENS, UI}{}\vspace{-9pt}
    %------------------------------------------------
    \entry{2009-2011}{Treasurer of Student Union (Heron) for postgraduates}{SENS, UI }{}\vspace{-9pt}
    %------------------------------------------------
    \entry{2011-2012}{Graduate student representative in Science Committee}{SENS, UI }{}\vspace{-9pt}
    %------------------------------------------------
    \entry{2009-2010}{Treasurer of BEST Reykjavik}{Board 
    of 
        European Students of Technology}{
        Participated in the BEST General Assembly on 
        behalf of BEST Reykjavik. \\
        Helped organize two BEST academic courses at 
        UI, where we housed and entertained 20 
        European students over a course of a week. 
        }
    %------------------------------------------------
    \entry{2006-2007}{President of Student union (Stigull) for undergraduates in Mathematics and Physics}{SENS, UI}{}
    %------------------------------------------------
\end{entrylist}

\clearpage
\section{publications}
%-------------------------------------------------------------------------------
%	PUBLICATIONS SECTION
%-------------------------------------------------------------------------------

{Literature available on 
\href{https://www.researchgate.net/profile/Helga_Ingimundardottir}{{\boldfont 
Research Gate}}}
\nocite{*}

%\printbibliography[type=misc, title={asdf}, heading=subbibliography]
\printbibsection{book}{thesis} 
\printbibsection{article}{article in peer-reviewed journals} 
\newpage
\printbibsection{inproceedings}{international peer-reviewed conferences/proceedings} 
\printbibliography[type=misc, title={seminars}, heading=subbibliography]


%-------------------------------------------------------------------------------

\end{document}
