\section{communication skills}
\begin{entrylist}
\entry{2024}{Newspaper Interview}{Morgunblaðið, Reykavík, Iceland}{
The article in the national newspaper highlights our innovative work where three students I am
supervising are hacking and upgrading a 90s knitting machine to modern-day standards so it can
run autonomously. This project is part of the \emph{HiDef Textiles} initiative, which combines
sustainable technology, textile innovation, STEAM education, AI, and IoT.
URL: \url{https://www.mbl.is/frettir/innlent/2024/06/02/sjalfvirknivaedd_prjonavel_i_bigerd/}
}
\entry{2024}{Oral Presentation}{The Icelandic Computer Society (SKÝ), Reykjavík, Iceland}
{People are an indispensable part of software development, where creativity, expertise, and
communication skills play a key role. In this event on \emph{software development and people}, I
discussed the common issue of lack of transparency and inadequate communication between teams.
Drawing from my experiences in software development across various industries, I provided
examples of how clear procedures, good oversight, and timely communication enhance efficiency and
improve everyone's experience.
URL: \url{https://www.sky.is/allir-vidburdir/2939-2024-hugbunadargerd-0320}}
\entry{2023}{Oral Presentation}{Haustráðstefna Advania, Reykjavík, Iceland}
{Presented an enhanced version of \emph{Pushing Boundaries: A Data-Driven Dive into
`Legend of the Ice People'} at Harpa on September 8th, 2023.
The presentation explored the synergy between literature and data science, extracting insights
from various formats of Margit Sandemo's series. It showcased the transformative power of data in
unanticipated domains, using datasets from the Icelandic book releases and Storytel audiobooks.
A recording is available on \href{https://www.youtube.com/watch?v=vyHsq-1FUHY}{YouTube}.}
\entry{2023}{Oral Presentation}{6th Reyjavík Data Beers, Reykjavík, Iceland}
{Delivered a talk titled \emph{The Legend of the Cat People: An Engineering Perspective on the
'80s Chick-Lit Sensation by Margit Sandemo (with Cat Memes)} at Databeers semi-formal seminar on
June 15, 2023, at Orkuveita Reykjavíkur. This talk highlighted data analysis from the
\emph{Legend of the Ice People} dataset, featuring information about the Icelandic book releases
from the 1980s and corresponding 2017 Storytel audiobooks. The presentation was based on insights
from my podcast \emph{ÍSKISUR} on Storytel.
All data and code is shared on \href{https://github.com/tungufoss/iskisur-rek-data-beers}{GitHub}.}
\entry{2022}{Panelist}{3rd European Language Resource Coordination (ELRC) workshop in Iceland}{
As an invited panelist at the \href{https://lr-coordination.eu/iceland3rd}{third Icelandic ELRC workshop}, I discussed the impact of Language Technology and AI on the Icelandic language with other developers, integrators, and users. We explored the potential of Language Technology to transform digital interactions in both private and public sectors and shared our experiences and perspectives on its current status and future prospects. The discussion was engaging and provided valuable insights into the role of Language Technology in shaping our multilingual future.}
%----------------------------------
\entry{2017-20}{Podcast Host, \href{https://www.storytel.com/is/is/series/28703-Iskisur}{ÍSKISUR}}{Alvarpið \& Storytel}{
I, along with two of my friends, co-hosted an Icelandic podcast where we read all 47 books in The Legend of the Ice People series by Margit Sandemo. Additionally, I curated a segment on Internet cats at the end of each episode. The podcast was originally published by Alvarpið from 2017-18, but was later moved to Storytel Iceland in March 2019.}
%----------------------------------
\entry{2016-17}{Recurring Podcast Guest, \href{https://taeknivarpid.is/}{Tæknivarpið}}{Hlaðvarp
Kjarnans}{As a frequent guest on Tæknivarpið, an Icelandic podcast dedicated to the newest
technology and devices, I had the chance to discuss various topics including the latest iPhones,
	Star Wars, and gender representation in IT. Through the podcast, I had the opportunity to
	share my insights and engage in stimulating conversations about the impact of technology on society.
}
%------------------------------------------------
\entry
{2016}
{Oral Presentation}
{Ph.D. defence 30th of June at Háskóla Íslands, Reykjavík, Iceland.}
{Presented my Ph.D. thesis \citetitle{phd-thesis}. 
Opponents: Prof. Edmund Burke and Prof. Kate Smith-Miles.}
%------------------------------------------------
\end{entrylist}
\begin{entrylist}
\entry{2016}{Founding Workshop}{Samtök kvenna í vísindum (SKVÍS)}{ % Feb 2016
	I was fortunate enough to be a part of the founding workshop of SKVÍS, the Association of Women in Science. 
	The workshop united women from various scientific fields to work towards specific goals aimed at promoting gender equality in science. 
	My role in the workshop involved collaborating with other participants to develop strategies for increasing the visibility of women experts in the media, challenging the prevailing bias towards male experts. 
	Overall, it was an incredibly inspiring and productive event that helped to lay the foundation for SKVÍS as an organization dedicated to supporting women in science.
}
%------------------------------------------------
\entry
{2015}
{Oral Presentation}
{9th Int'l Conference on Learning and Intelligent Optimization (LION9)}
{Presented the paper \citetitle{phd-conf-lion9} in Lille, France.}
%------------------------------------------------
\entry
{2012}
{Oral Presentation}
{6th Int'l Conference on Learning and Intelligent Optimization (LION6)}
{Presented the paper \citetitle{phd-conf-lion6} in Paris,  France.}
%------------------------------------------------
\entry
{2011}
{Oral Presentation}
{11th Int'l Conference on Intelligent Systems Design \& Applications (ISDA)}
{Presented the paper  \citetitle{phd-conf-isda11} in Cordoba, Spain.}
%------------------------------------------------
\entry
{2010}
{Oral Presentation}
{5th Int'l Conference on Learning and Intelligent Optimization (LION5)}
{Presented the paper \citetitle{phd-conf-lion5} in Rome, Italy}
%------------------------------------------------
\entry
{2010}
{Invited Speaker}
{Silisian University, Gliwice, Poland}
{I was invited by Prof. Waldemar Grzechca to come to Politechnika Śląska in Gliwice, Poland, as part of the Erasmus+ Teaching Staff Mobility Progamme. During my stay, I presented my Ph.D. research to the Faculty of Automation, Electronics, and Computer Science.}
%------------------------------------------------
\entry
{2009}
{Invited Speaker}
{University of Valenciennes and Hainaut-Cambresis, Valenciennes, France}
{As part of the collaboration with Sylvan Lalot, I presented the my Masters research to the faculty at the ENSIAME department at UVHC.}
%------------------------------------------------
\end{entrylist}
\begin{entrylist}
\entry
{2009}
{Poster}
{11th Int'l Conference on Heat Exchanger Fouling and Cleaning}
{Presented the paper \citetitle{msc-conf-schladming} in Schladming, Austria.}
%------------------------------------------------
\end{entrylist}
